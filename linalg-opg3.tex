\documentclass[11pt]{article}
\usepackage[utf8]{inputenc}
\usepackage[T1]{fontenc}
\usepackage[usenames]{color}
\usepackage[parfill]{parskip}
\usepackage[danish,english]{babel}
\usepackage[a4paper, hmargin={2.8cm, 2.8cm}, vmargin={2.5cm, 2.5cm}]{geometry}
\usepackage[babel, lille, farve, nat, en]{/Users/ku-forside/ku-forside}
\usepackage{lmodern}
\usepackage{amsmath}
\usepackage{amsfonts}
\usepackage{amssymb}
\usepackage{mathdots}
\usepackage{graphicx}
\usepackage{physics}
\usepackage{booktabs}
\usepackage{listings}
\usepackage{siunitx}
\usepackage[normalem]{ulem}
\useunder{\uline}{\ul}{}
\usepackage{gensymb}
\usepackage{lastpage}
\usepackage{fancyhdr}
\usepackage{blindtext}
\usepackage{bm,upgreek}
\pagestyle{fancy}
\fancypagestyle{firststyle}
{
   \fancyhf{}
   \fancyhead{}
   \renewcommand{\headrulewidth}{0pt}
   \fancyfoot[R]{Page~\thepage\ of~\pageref{LastPage}}
   \fancyfoot[L]{\today}
}
\lhead{Thorsten Bæk}
\chead{Linear Algebra weekly assignment 3}
\rhead{kmg386}
\rfoot{Page~\thepage\ of~\pageref{LastPage}}
\cfoot{ }
\lfoot{\today}

\titel{Linear Algebra weekly assignment 3}
\undertitel{Linear Algebra}
\opgave{Assignment}
\forfatter{\small Thorsten Bæk}
\dato{\today}
\vejleder{}
\begin{document}

\maketitle
\thispagestyle{firststyle}
%\abstract{}

\section*{3.1}

\subsection{(a)}

The matrix \(\underline{\underline{{A_n}}}\) has the following form

\begin{equation}
  \underline{\underline{{A_n}}}=
  \begin{pmatrix}
   0      & \cdots  & \cdots      & 0 & 1_{1n} \\
  \vdots  &        &         & \iddots &  0 \\
  \vdots  &         & 1_{ji}  &          & \vdots  \\
   0      & \iddots &         &   &  \vdots \\
   1_{m1} & 0  & \cdots  & \cdots &  0
\end{pmatrix}
\end{equation}

\subsection{(b)}

Input in Maple

\begin{verbatim}
  for i to 10 do Determinant(FlipDimension(IdentityMatrix(i), 1)) end do
\end{verbatim}

Output:

\begin{equation}
1, -1, -1, 1, 1, -1, -1, 1, 1, -1
\end{equation}

The determinant is always 1 or -1, depending on \(n\).
It seems to change sign each time \(n\) passes an even number.

\subsection{(c)}

The pattern in the determinants sign can be explained in a few ways.
One is when using row operations - when ever you switch rows, you change the sign of the determinant.
This creates a pattern, as we see. If you divice \(n\) with 2 and get an even number, you get an even determinant, ignoring the added \(\frac{1}{2}\) for every second number in the sequence, since it represents the row in "the middle" on which you don't make operations.
If you find the determinant by expansion, you see that for each \(n\) you multiply the determinant of \(n-1\) with 1 minding whether or not the 1 in the added row is in a position that makes it negative or positive.
This gives you a pattern of the determinant changing for every second step in the \(n\) sequence.
\newline
I would prefer using row operations to explain the sequence, since it is easy to see how times you need to switch rows to get the unit matrix.
Combining this with the knowledge of the rule for switching sign of the determinant for each row switch, it easy to see the connection.

\subsection{(d)}

Each time you switch rows, you invert the orientation of the function.
So, \(f_n\) does not keep the orientation for each \(n\) but switch orientation with the same sequence as the determinant switches sign.

\section{3.2(iii)}

\subsection{(a)}

\( \bm{\Psi}_3 \):

\begin{equation}
  \begin{split}
    \bm{\Psi}_3(e_1,e_2,e_3) = \frac{1}{\sqrt{6}} ( \psi_1(e_1) \psi_2(e_2) \psi_3(e_3) + \psi_2(e_1) \psi_3(e_2) \psi_1(e_3) + \psi_3(e_3) \psi_1(e_2) \psi_2(e_3) \\
    - \psi_1(e_3) \psi_2(e_2) \psi_3(e_1) - \psi_2(e_3) \psi_3(e_2) \psi_1(e_1) - \psi_3(e_3) \psi_1(e_2) \psi_2(e_1) )
  \end{split}
\end{equation}

\subsection{(b)}

Since \(\bm{\Psi}_N(e_1,e_2,...,e_N)\) is defined as \(\frac{1}{\sqrt{N!}}\) times the determinant of a matrix, switching two coloums or rows will change the sign of \(\bm{\Psi}_N(e_1,e_2,...,e_N)\) as those operations changes the sign of a determinant.
See Sentence 3.4.3, NVP - Lineær Algebra.
This means that \(\bm{\Psi}_N(e_1,e_2,...,e_N)\) is asymmetric.

\subsection{(c)}

If \(\psi_i = \psi_j \) for \(i \neq j\) this means that two columns will be equal. As we saw in question 3.2(iii)(b), operations effecting a determinant of a matrix will affect the value of \(\bm{\Psi}_N(e_1,e_2,...,e_N)\) due to its definition.
Two equal columns or rows will result in the determinant being 0, which means \(\bm{\Psi}_N(e_1,e_2,...,e_N)  = 0 \). This is because you can subtract either coloumn from the other resulting in a coloums of zeroes.
Finding the determinant be expansion afterwards will allow you to multiply by all zeroes resulting in a determinant equal to zero.

\end{document}
